Im Rahmen des Praxisprojekts wird ein Minimum Viable Product
(\gls{MVP}) 
für eine Anwendung entwickelt, auf der sich Nutzer kennenlernen und miteinander interagieren können. Zielgruppe der Anwendung sind Spieler des Videospiels \textit{League of Legends}, welche nach neuen Mitspielern für gemeinsame Partien suchen.

Um die Anwendung im vollen Umfang nutzen zu können, muss sich der Nutzer ein Konto erstellen.
Dazu sind Angaben über Email-Addresse und gewünschtem Nutzernamen und Passwort notwendig.
Nach der Registrierung kann sich der Nutzer auf dem eigenen Profil beschreiben.
Denkbare Felder sind ein Freitext, ein Profilbild, das Alter, das Geschlecht, präferierte Spielpositionen und Helden sowie Lieblingsspielmodi.
Die Selbstbeschreibung soll anderen Nutzern helfen, einschätzen zu können, ob man zueinander passt.\\

%Suche
Andere Nutzer sollen über eine Suchfunktion findbar sein.
Neben einer Suche ohne Parameter, welche zufällige Nutzer anzeigt, soll eine Filterfunktion erstellt werden.
Diese erlaubt es, die Suche auf Nutzer mit den gewählten Eigenschaften, zum Beispiel einem bestimmten Geschlecht oder einer präferiterten Spielposition, einzugrenzen. \\

%Freunde
Das Befreunden mit anderen Nutzern soll ähnlich wie auf dem Datingportal \textit{Tinder} gelöst werden: Nutzern kann ein \textit{Like} (\enquote{mag ich}) gesendet werden.
Dies sorgt dafür, dass die suchende Person in der nächsten Suche der gesuchten Person oben angezeigt wird.
Wenn die gesuchte Person den \textit{Like} erwiedert, werden die beiden Nutzer befreundet.
Befreundete Nutzer sollen in der Lage sein, miteinander über einen Chat zu kommunizieren.
Dieser Chat erlaubt das Schreiben von Text nach \gls{UTF-8} Spezifikation,
dies ermöglicht das schicken von Emojis.\\

Fehlverhalten, wie zum Beispiel die Bewerbung von kostenpflichtigen Diensten sowie explizite Inhalte und aufdringliches Verhalten sind unerwünscht.
Nutzer sollen daher in der Lage sein, solches Verhalten zu melden.
Meldungen werden durch Moderatoren verfolgt und ggf. bestraft.
Auch soll das Blockieren von anderen Nutzern möglich sein.
Blockierte Nutzer sollen automatisch von der Freundesliste entfernt werden und werden in der Suche nicht mehr angezeigt.
Wenn ein Nutzer Fehlverhalten meldet, wird der gemeldete Nutzer automatisch blockiert.\\

Sollte ein Nutzer mit dem Produkt nicht zufrieden sein, steht es ihm frei, das Konto zu löschen.
Hier bietet es sich an, den ehemaligen Nutzer nach Verbesserungsvorschlägen zu fragen, um Makel ausbessern zu können.\\

Die Nutzer sollen in der Lage sein, sich ein Benutzerkonto erstellen, auf dem Profil Angaben über die eigene Person machen, andere Nutzer per Suchfunktion finden und Freundschaftsanfragen schicken zu können.
Personen, die miteinander befreundet sind, können im Chat Nachrichten austauschen.
Dadurch können zum Beispiel Termine zum gemeinsamen Spielen vereinbart, Kontaktdaten ausgetauscht und Freundschaften geschlossen werden.\\

%Auf der Startseite soll ein Interessent einen ersten Eindruck von der Anwendung erhalten, bevor dieser sich ein Konto erstellen muss.

Um viele Nutzer erreichen zu können, soll die Anwendung in Englisch verfügbar sein.
Durch  \gls{Responsive Webdesign}
soll die Anwendung Desktop und Mobil unterstützen können, dies erhöht die potenzielle Nutzerbasis.\\

Aus Softwaresicht sollte auf eine gute Dokumentation geachtet werden, um die Wartbarkeit des Projektes zu erhöhen und Kollaboratoren einen einfacheren Einstieg zu ermöglichen.
Um die Chance von erfolgreichen Brute-Force-Angriffen auf Passwörter zu verringern, sollen sinnvolle Anforderungen an das vom Nutzer gewählte Passwort gestellt werden.
Auch sollen die Passwörter ausreichend verschlüsselt werden, um die Auswirkungen von Datenbankangriffen zu verringern.\\

Da es sich bei dem Projekt um ein MVP handelt, sollen nur die nötigsten Grundfunktionen entwickelt werden.
Themen wie UI/UX und Finanzierung liegen daher nicht im Rahmen des Projekts.
Das Ziel ist, die Idee einer Kennenlern-Anwendung zu testen und zu prüfen, ob es sich lohnt, das Projekt weiterzuentwickeln. 