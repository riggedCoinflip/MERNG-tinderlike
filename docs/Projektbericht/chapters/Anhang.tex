%\appendix
%\chapter{Anexo I: Ejemplo de hoja técnica}
% Contenido del anexo I
%\chapter{Anexo II: Hoja de soluciones}
% Contenido del anexo II

\subsection{Beispiel einer Antwort auf die an den Server gesendete Anfrage }\label{subsec_UabsAnhang}
\begin{lstlisting}
    name: "gilo2754"
    aboutMe: "I like video code video games too"
    avatar: "https://lol-friendfinder-dev.s3.eu-central-1.amazonaws.com/avatars/a028fd00-e767-4dcb-912b-b4828dbe52e7.jpg"
    dateOfBirth: "1990-06-01T00:00:00.000Z"
    gender: "male"
    ingameRole: (2) ["Fill", "Bot"]
    languages: (2) ["bm", "de"]
    friends:(5) ["60eb61b33a2481451c1cd7ac", "60eb61b33a2481451c1cd7b0", "60eb61b33a2481451c1cd7b1", "60eb61b33a2481451c1cd7b4", "60eb61b33a2481451c1cd7b7"]
    blocked: (1) ["60eea5eed7e23b5ef08f68af"]
\end{lstlisting}


\subsection{Beispiel eines End-to-End Testfälles}\label{subsec_UabsAnhang}
\begin{lstlisting}
    //in constants.js sind Standard-Testdaten
    import * as c from "../constants.js"
    
    const qtyLanguages = 101
    const qtyGenders = 8
    
    describe("Check if all profile elements have been rendered", () => {
      it("Login with existing account", () => {
        cy.typeLogin(c.email, c.password)
      })
    
      it("Username was rendered", () => {
        cy.get('#username').should("be.visible")
      })
    
      it("Gender selection was rendered", () => {
        cy.get('#dropdown-gender').should(
          "be.visible")
      })
    
    //Der folgende Test prueft, ob eine Mindestanzahl von Elementen angezeigt wird  
    it(`Gender options is not empty and >= to ${qtyGenders}`, () => {
      cy.get('#dropdown-gender').click()
      cy.get(`.dropdown-menu > :nth-child(${qtyGenders})`).should(
          "be.visible")
      //close the dropdown
      cy.get('#dropdown-gender').click()
    
      })
    ...
    })
    
    \end{lstlisting}
% #3
\subsection{Beispiel eines Commands bei Cypress}\label{subsec_UabsAnhang}
\begin{lstlisting}
    Cypress.Commands.add("typeLogin", (email, password) => {
        cy.visit("/login")
    
        cy.get("[id=email-input]").type(email).should("have.value", email)
    
        cy.get("[id=password-input]").type(password).should("have.value", password)
    
        cy.get("[id=btn-submit]").click()
    })
\end{lstlisting}
Es handelt sich in diesem Fall, um eine benutzerdefinierte Funktion, die zwei Parameter, E-Mail und Passwort, erhält. 
Beide Parameter werden in die entsprechenden Eingabefelder eingetragen, abschließend wird die Eingabetaste gedrückt.
Diese Funktion kann in jedem anderen Testfall aufgerufen werden, ohne dass sie importiert werden muss.

% #4
\subsection{Code-Auszug zum Prüfen und Hochladen von einem Bild mit Axios}\label{subsec_UabsAnhang}
\begin{lstlisting}
    function disableBtn() {
        const uploadBtn = document.getElementById("uploadBtn")
        uploadBtn.disabled = true
        uploadBtn.style.background = "#000000"
      }
    
      function fileUploadHandler() {
        errored ? 
         disableBtn()
         /*
        Tritt ein Fehler auf, wird die Abfrage nicht gesendet und der Knopf deaktiviert 
        Ein Fehler tritt auf, wenn die Datei zu gross ist oder oder wenn sie ein ungueltiges  Format hat
        */
        :
        console.log("uploading pic...", file?.name)
        const fd = new FormData()
        /* avatar ist der Name der Datei, die hochgeladen wird
          file ist der zu sendende  Wert */
        fd.append("avatar", file)  
    
        axios
          .post(urlAvatar, fd, {
            headers: {
              "x-auth-token": TOKEN,
            },
          })
          .then((res) => {
            setState((state) => ({ ...state, avatar: res?.data?.location }))    
            //In location finden wir die URL fuer das gerade hochgeladende Bild 
          })
      }\end{lstlisting}

%#5
\subsection{Definition des Providers für das Teilen von Werten mit dem Consumer }\label{subsec_UabsAnhang}
Dieser Code ermöglicht, die Werte innerhalb von dem Context zu verwenden.
\begin{lstlisting}
    import { React, useState, useEffect, createContext } from "react"

    export const GlobalContext = createContext()
    export default function App() {
        const [token, setToken] = useState()
        const [state, setState] = useState(null)
        
        ...
   
        return (
            <GlobalContext.Provider value={{ token, setToken, state, setState, refetch }}>
              <>
                <MyNavbar />
        
                <Switch>
                  <Route exact path="/" component={Home} />
                  <Route path="/login" component={() => <Login />} />
                  <Route path="/signup" component={SignUp} />
                  <Route exact path="/users" component={() => <Users />} />
                  <Route exact path="/profile" component={() => <Profile />} />
                  <Route exact path="/chat" component={() => <Chat />} />
                  <Route exact path="/ChatMessage" component={() => <ChatMessage />} />
        
                  <Route component={NotFound} />
                </Switch>
              </>
            </GlobalContext.Provider>
          )
        }
//Code-Auszug in frontend/src/App.js
 \end{lstlisting}

%#5
\subsection{GlobalContext, um die Freunde des angemeldeten Benutzers zu zeigen. }\label{subsec_UabsAnhang}
Nachfolgend wird gezeigt, wie GlobalContext in der Komponente Chat.js verwendet wurde, um die Freunde des angemeldeten Benutzers zu ermitteln.

Die Kennungen der Freunde waren notwendig, um die Kommunikation zu ermöglichen und die zwischen den Freunden ausgetauschten Nachrichten zu laden.
\begin{lstlisting}
    import { useContext, useEffect, useState, React } from "react"
    import { GlobalContext } from "../App"
    //mithilfe von Destrukturierende Zuweisung wurden neue Variablen und Funktionen definiert
    const { token, state, setState, refetch } = useContext(GlobalContext)

    ...

    return !token ? (

    ...

        <div className="user-many">
        {state?.friends &&
          state?.friends?.map((item, index) => {
            return (

//Die Komponente FriendList gibt den Namen und den Avatar des Benutzers auf der Grundlage der eingegebenen ID zurueck.
              <FriendList
                setUserID={setUserID}
                setUserNameChat={setUserNameChat}
                setChatAvatar={setChatAvatar}
                userId={item.user}
                searchUser={searchUser}
                setSearchUser={setSearchUser}
                key={index + 1}
              />
            )
          })}
      </div>
    )
  ...
//Code-Auszug in frontend/src/Chat.js
\end{lstlisting}

\subsection{Links zu den verwendeten Technologien und Werkzeugen}\label{subsec_UabsAnhang}
NodeJS
\\
React
\\
Bootstrap
\\
GitHub
\\
Cypress
\\
VSC
\\
Heroku
\\
LaTeX
\\
Cypress

\subsection{Aufwandsverteilung}\label{subsec_UabsAnhang}
Hier zeigen wir, wie die Aufgaben unter den Autoren des Projekts verteilt wurden.
TABELLE/Bild KOMMT...