%%% MUSTER %%%
%\newglossaryentry{}
%{name=, description={}}

%%%%%%%%%%%%
%%% TIMO %%%
%%%%%%%%%%%%
\newglossaryentry{MVP}
{
    name=MVP,
    description={Minimum Viable Product, erste funktionsfähige Version des Produkts; dient dazu zu prüfen, ob es Sinn macht, das Produkt weiterzuentwickeln und ggf. anzupassen}
}


\newglossaryentry{UTF-8}
{
    name=UTF-8,
    description={Weit verbreitete Kodierung für Unicode-Zeichen, sehr beliebt für Webseiten}
}

\newglossaryentry{Responsive Webdesign}
{
    name=Responsive Webdesign,
    description={Die Webseite wird so aufgebaut, dass sie auf möglichst vielen Endgeräten mit verschiedenen Bildschirmgrößen und Eingabemethoden (Maus, Touchscreen) skaliert wird und bedienbar ist.}
}

\newglossaryentry{DQL}
{
    name=DQL,
    description={Data Query Language, Datenabfragesprache für SQL}
}
\textbf{DQL:}\\
Data Query Language, Datenabfragesprache für SQL
\\\\
\textbf{Destrukturierende Zuweisung:}\\
Die destrukturierende Zuweisung ermöglicht es, Daten aus Arrays oder Objekten zu extrahieren, und zwar mit Hilfe einer Syntax, die der Konstruktion von Array- und Objekt-Literalen nachempfunden ist.
%\url{https://developer.mozilla.org/de/docs/Web/JavaScript/Reference/Operators/Destructuring_assignment}
\\\\
\textbf{MVP:}\\
Minimum Viable Product, erste funktionsfähige Version des Produkts; dient dazu zu prüfen, ob es Sinn macht, das Produkt weiterzuentwickeln und ggf. anzupassen.
\\\\
\textbf{JSX:}\\
Es heißt JSX und ist eine Syntaxerweiterung für JavaScript. JSX erinnert vielleicht an eine Template-Sprache, aber es verfügt über die volle Leistungsfähigkeit von JavaScript. JSX erzeugt React-"Elemente"
\\\\
\textbf{Over-Fetching:}\\
Empfang von überschüssigen Daten durch eine Abfrage.
%Reference https://jwt.io/
\\\\
\textbf{undefined:}\\
Eine Variable, der kein Wert zugewiesen wurde oder die überhaupt nicht deklariert wurde (nicht deklariert, existiert nicht), ist undefiniert. Eine Methode oder Anweisung gibt auch undefiniert zurück, wenn der ausgewerteten Variablen kein Wert zugewiesen wurde. Eine Funktion gibt undefiniert zurück, wenn kein Wert zurückgegeben wurde.
\\\\
\textbf{Responsive Webdesign:}\\
Die Webseite wird so aufgebaut, dass sie auf möglichst vielen Endgeräten mit verschiedenen Bildschirmgrößen und Eingabemethoden (Maus, Touchscreen) skaliert wird und bedienbar ist.
\\\\
\textbf{UTF-8:}\\
Weit verbreitete Kodierung für Unicode-Zeichen, sehr beliebt für Webseiten.\\\\
\textbf{Web Token:}\\
JSON-Web-Tokens sind eine dem Industriestandard RFC 7519 entsprechende Methode zur sicheren Darstellung von Forderungen zwischen zwei Parteien.
\\\\
\section*{Akronyme:}
{AWS}: {Amazon Web Services}\\
{DOM}: {Document Object Model}\\
{API}: {Application Programming Interface}\\
{AJAX}: {Asynchronous JavaScript And XML}\\
