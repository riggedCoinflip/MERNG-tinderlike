WORK IN PROGRESS...

\subsection{Einführung in das Thema (Motivation, zentrale Begriffe etc.)}
\subsection{Hinführung zu den Ergebnissen}
\subsection{Ggf. Angabe des Schwerpunktes}
\subsection{Ggf. Einschränkungen darlegen}
\subsection{Problemstellung}
\subsection{Zielstellung der Arbeit}
\subsection{Fragestellung der Arbeit}

\subsection{Struktur der Arbeit}
Dieser Projektbericht ist in folgenden Kapitel unterteilt:
\textbf{Kapitel 2}
\\\\
\textbf{Kapitel 3}
\\\\
%Benutzeroberfläche
\textbf{Kapitel 4?} erläutert, welche JavaScript-Frameworks zu Beginn des Projekts berücksichtigt wurden. Er gibt einen Überblick über die Faktoren, die zu beachten sind, wenn man sich für React entscheidet, und zeigt schließlich, wie die Daten mit Hilfe der React JavaScript-Bibliothek und ihrer Hooks für den Endbenutzer visualisiert werden.
\\\\
%Serverabfragen 
\textbf{Kapitel 5?} geht es um die Interaktion zwischen dem Client und dem Server. 
\\\\
Anhand von zwei Beispielen wird gezeigt, wie Lese- und Schreibabfragen mit Hilfe von GraphQL und ApolloClient durchgeführt wurden.
\\\\
%Qualitätssicherung
\textbf{Kapitel 6?} zeigt die Relevanz von Testfällen. Sie zeigt auch die Vorteile automatisierter und eingebetteter Testfällen in der Entwicklungsumgebung.
\\