WORK IN PROGRESS...

\subsection*{Einführung in das Thema }%(Motivation, zentrale Begriffe etc.)}
\addcontentsline{toc}{subsection}{Einführung in das Thema}
%\subsection*{Hinführung zu den Ergebnissen}
%\addcontentsline{toc}{subsection}{Hinführung zu den Ergebnissen} 
%\subsection*{Ggf. Angabe des Schwerpunktes}
%\addcontentsline{toc}{subsection}{Angabe des Schwerpunktes} 
\subsection*{Ggf. Einschränkungen darlegen}
%\addcontentsline{toc}{subsection}{Einschränkungen darlegen} 
\subsection*{Problemstellung}
\addcontentsline{toc}{subsection}{Problemstellung} 
\subsection*{Zielstellung der Arbeit}
\addcontentsline{toc}{subsection}{Zielstellung der Arbeit} 
\subsection*{Fragestellung der Arbeit}
\addcontentsline{toc}{subsection}{Fragestellung der Arbeit} 

\subsubsection*{Struktur der Arbeit}
\addcontentsline{toc}{subsection}{Struktur der Arbeit} % Manuellen Eintrag im Inhaltsverzeichnis erzeugen
Dieser Projektbericht ist in folgenden Kapitel unterteilt:\\
Im \textbf{Kapitel \ref{kap_Technologieinfrastruktur}}
\\\\
\textbf{Kapitel \ref{kap_Datenbank}}
\\\\
Kapitel \ref{kap_Backend} 
TODO
\\\\
%Serverabfragen 
Kapitel \ref{kap_Schnittstelle} geht es um die Interaktion zwischen dem Client und dem Server. 
\\\\
%Anhand von zwei Beispielen wird gezeigt, wie Lese- und Schreibabfragen mit Hilfe von GraphQL und ApolloClient durchgeführt wurden.
\\\\
Kapitel \ref{kap_Frontend} erläutert, welche JavaScript-Frameworks zu Beginn des Projekts berücksichtigt wurden. Es gibt einen Überblick über die Faktoren, die zu beachten sind, wenn man sich für React entscheidet, und zeigt schließlich, wie die Daten mit Hilfe der React JavaScript-Bibliothek und ihrer Hooks für den Endbenutzer visualisiert werden.
\\\\
%Qualitätssicherung
Kapitel \ref{kap_QS} zeigt die Relevanz von Testfällen. Sie zeigt auch die Vorteile automatisierter und eingebetteter Testfällen in der Entwicklungsumgebung.

