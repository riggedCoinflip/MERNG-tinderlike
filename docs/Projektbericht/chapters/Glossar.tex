%\newglossaryentry{kiln}
%{
 % name=kiln,
  %description={German: Brennofen (m.);\\Français: fourneau (m.)},
  %plural=kilns
%}

%Make Glossary properly...
%\acrodef{VB}{Visula Basic}

\textbf{Hooks:}\\
...
\textbf{Framework:}\\
...
\textbf{JSX:}\\
Es heißt JSX und ist eine Syntaxerweiterung für JavaScript. Wir empfehlen, sie mit React zu verwenden, um zu beschreiben, wie die Benutzeroberfläche aussehen soll. JSX erinnert vielleicht an eine Template-Sprache, aber es verfügt über die volle Leistungsfähigkeit von JavaScript. JSX erzeugt React-"Elemente"

\textbf{Over-Fetching:}\\
Empfang von überschüssigen Daten durch eine Abfrage.
%Reference https://jwt.io/

\textbf{Web Token:}\\
JSON-Web-Tokens sind eine dem Industriestandard RFC 7519 entsprechende Methode zur sicheren Darstellung von Forderungen zwischen zwei Parteien.

\textbf{undefined:}\\
Eine Variable, der kein Wert zugewiesen wurde oder die überhaupt nicht deklariert wurde (nicht deklariert, existiert nicht), ist undefiniert. Eine Methode oder Anweisung gibt auch undefiniert zurück, wenn der ausgewerteten Variablen kein Wert zugewiesen wurde. Eine Funktion gibt undefiniert zurück, wenn kein Wert zurückgegeben wurde.

\textbf{FormData:}\\
Die FormData-Schnittstelle bietet eine einfache Möglichkeit, eine Reihe von Schlüssel/Wert-Paaren zu erstellen, die die Felder eines Formulars und ihre Werte darstellen und mit der XMLHttpRequest.send()-Methode einfach gesendet werden können.

\textbf{componentDidMount:}\\
componentDidMount() wird unmittelbar nachdem eine Komponente (Einfügen in den Baum) montiert. Die Initialisierung, die DOM-Knoten erfordert, sollte hier erfolgen. Wenn Daten von einem Endpunkt geladen werden müssen, ist dies ein guter Ort, um die Netzwerkanfrage zu instanziieren.

Diese Methode ist ein guter Ort, um Abonnements einzurichten. Wenn das der Fall ist, sollte es nicht vergessen werden, sich in componentWillUnmount() abzumelden.


\textbf{componentDidUpdate:}\\
componentDidUpdate() wird unmittelbar nach der Aktualisierung aufgerufen. Diese Methode wird beim ersten Rendering nicht aufgerufen.

\textbf{componentWillUnmount:}\\
componentWillUnmount() wird aufgerufen, unmittelbar bevor eine Komponente demontiert und zerstört wird. Man sollte in dieser Methode alle notwendigen Bereinigungen durchgeführen, wie z. B. das Ungültigmachen von Zeitgebern, das Abbrechen von Netzwerkanforderungen oder das Aufräumen von Abonnements, die in componentDidMount() erstellt wurden.




\textbf{Akronyme:}\\

{AWS}{Amazon Web Services}\\
{DOM}{Document Object Model}\\
{API}{Application Programming Interface}\\

