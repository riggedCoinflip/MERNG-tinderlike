\paragraph{}
Unsere Plattform hat das Ziel, dass Spieler sich gegenseitig finden können, sich vernetzen können. Spielern werden dazu andere Spieler angezeigt und können entscheiden, ob sie mit diesen in Kontakt treten wollen. 

Auf der Datenbank müssen diese Angaben entsprechend gespeichert werden. Wir nennen die dadurch erstellten Dokumente 'likes' bzw. 'dislikes'. Die Datenbankeinträge haben entsprechend folgende Felder:

\begin{center}
    \begin{tabular}{ |p{0.15\linewidth}|p{0.64\linewidth}| } 
     \hline
     Feld & Beschreibung \\ 
     \hline
     Anfragender & NutzerID des Nutzers, der die Entscheidung getroffen hat \\
     Angefragter & NutzerID des Nutzers, über den die Entscheidung getroffen wurde \\
     Status & Entscheidung, kann 'liked' oder 'disliked' sein. \\
     \hline
    \end{tabular}
\end{center}

Jedes Mal, wenn ein Like in die Datenbank eingetragen wird, wird geprüft, ob es bereits ein Gegenpaar gibt. Ist dies der Fall, werden die beiden Nutzer auf die Freundesliste des jeweils anderen hinzugefügt.

Beispiel: Alice hat Bob einen like gegeben. Bob gibt im Anschluss Alice einen like. Da es bereits einen Datensatz des Gegenpaar gibt (Alice mag Bob), sind die beiden nun befreundet.


Bei der Suche werden nur "neue" Nutzer angezeigt, das bedeutet, nur Nutzer, die vom aktiven Nutzer nicht bereits geliked, disliked oder geblockt wurden und sich auch nicht bereits in der Freundesliste befinden.