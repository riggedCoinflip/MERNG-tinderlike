    % INFO: Biblatex -Ausgabe des  
  % Literaturverzeichnisses (Beispiele):   
  % - \printbibliography => Ausgabe ALLER 
  %   Einträge
  % - \printbibliography[nottype=online]
  %   => Ausgabe der Einträge, bis auf die
  %      "Online"-Einträge
  % - \printbibliography[type=online]     
  %   => Ausgabe nur der "Online"-Einträge  
   %\printbibliography

  
  % Literaturverzeichnis
   % INFO: Referenzieren auf das Literaturverzeichnis:
   %
   % Befehl: \cite{refmarke}
   % 
   % "refmarke" ist die Angabe in den geschweiften Klammern bei 
   % \bibitem[]{refmarke}. 
   \newpage
    \thispagestyle{empty}
   \section{Quellenverzeichnis}
     \subsection{Literatur}
     \renewcommand{\refname}{} % Literaturverzeichnis ohne Bezeichnung
     % Literaturverzeichnis
     \begin{thebibliography}{SW11} % 2. {...} => Hier die größte /breiteste Nummer (z.B. 99) oder Kurzbeleg angeben.
       \bibitem{SW11} Stickel-Wolf, Christine; Wolf, Joachim (2011): Wissenschaftliches Lernen und Lerntechniken. Erfolgreich studieren–-gewusst wie!. Wiesbaden: Gabler. 
        % TODO cite correctly
       \bibitem{PG01} Seite 51 Zeile 5-6 https://www.researchgate.net/profile/Ciprian-Octavian-Truica/publication/264416935_Asynchronous_Replication_in_Microsoft_SQL_Server_PostgreSQL_and_MySQL/links/53dbe6160cf216e4210c0375/Asynchronous-Replication-in-Microsoft-SQL-Server-PostgreSQL-and-MySQL.pdf
       \bibitem{DB01} Scalability Databases https://ieeexplore.ieee.org/abstract/document/7369245
      \end{thebibliography} 
          
     \subsection{Internetquellen}
     \begin{thebibliography}{HR08} % 2. {...} => Hier die größte/breiteste Nummer (z.B. 99) oder Kurzbeleg angeben.
       \bibitem{BBoJ}Bertelsmeier, Birgit (o. J.): Tipps zum Schrei\-b\-en ei\-n\-er Ab\-sch\-luss\-ar\-beit. Fach\-hoch\-schu\-le Köln-Campus Gummersbach, Institut für Informatik. \url{http://lwibs01.gm.fh-koeln.de/blogs/bertelsmeier/files/2008/05/abschlussarbeitsbetreuung.pdf} (29.10.2013).
        \bibitem{HR08} Halfmann, Marion; Rühmann, Hans (2008): Merkblatt zur Anfertigung von Projekt-, Bachelor-, Master- und Diplomarbeiten der Fakultät 10. Fachhochschule Köln-Campus Gummersbach.\url{http://www.f10.fh-koeln.de/imperia/md/content/pdfs/studium/tipps/anleitungda270108.pdf} (29.10.2013).
        \bibitem{V01} Offizielle Vue-Website: Vergleich zwischen Vue, React und Angular. \url{https://vuejs.org/v2/guide/comparison.html#Preact-and-Other-React-Like-Libraries} (unbekannte Veröffentlichung).
        \bibitem{R01}Offizielle React-Website: React Hooks. \url{https://reactjs.org/docs/hooks-faq.html#which-versions-of-react-include-hooks}
        \bibitem{A01}Offizielle Website Apollo für React. \url{https://www.apollographql.com/docs/react/}
        \bibitem{SO01}StackOverFlow: Developer Survey 2021. \url{https://insights.stackoverflow.com/survey/2021#section-most-popular-technologies-web-frameworks}
        \bibitem{EE1}Elad Elrom: React and Libraries. \url{https://link.springer.com/content/pdf/10.1007%2F978-1-4842-6696-0.pdf}
        \bibitem{SS1}Stoyan Stefanov: Durchstarten mit React. \url{https://content-select.com/media/moz_viewer/5d5fc360-478c-4038-ac17-246bb0dd2d03/language:de}
        \bibitem{RH1}Red Hat: Was ist GraphQL? \url{https://www.redhat.com/de/topics/api/what-is-graphql}
        \bibitem{PM1}Postman: 2020 State of the API Report \url{https://www.postman.com/state-of-api/the-future-of-apis/#the-future-of-apis}
        \bibitem{AX1}Offizielle Website Axios \url{https://axios-http.com/}

        \bibitem{PG11}Offizielle Webseite PostgreSQL \url{https://www.postgresql.org/}
        \bibitem{PG12}PostgreSQL: Warum sollte man PostgreSQL verwenden? \url{https://www.postgresql.org/about/}
        \bibitem{PG13}Stackshare: Wer benutzt PostgreSQL? \url{https://stackshare.io/postgresql}

        \bibitem{MG1}Offizielle Webseite MongoDB \url{https://www.mongodb.com/de-de}
        \bibitem{MG2}Sharding \url{https://docs.mongodb.com/manual/sharding/}
        \bibitem{MG3}Replizierung \url{https://docs.mongodb.com/manual/replication/}
        \bibitem{MG2}NoSQL Erklärt \url{https://www.mongodb.com/de-de/nosql-explained}
        \bibitem{JSON1}Einführung in JSON \url{https://www.json.org/json-de.html}

        \bibitem{12FA1}The Twelve-Factor-App: X.Dev-Prod-Vergleichbarkeit \url{https://12factor.net/de/}


     \end{thebibliography}