% Welche Datenbank wurde verwendet und warum,...
\paragraph{}
Nutzer unserer Plattform erstellen sich ein Profil, auf dem sie etliche Daten angeben können. Diese Daten helfen anderen Nutzern bei der Kontaktsuche. Sobald sich 2 Nutzer gefunden haben, können sie miteinander schreiben und die Texte auch später noch lesen. Um all dies zu ermöglichen und verwalten zu können, benötigt es Infrastruktur in Form von einer Datenbank.

\paragraph{}
In der näheren Auswahl standen MongoDB, eine "universelle, dokumentbasierte, verteilte Datenbank für die moderne Anwendungsentwicklung und die Cloud" [https://www.mongodb.com/de-de] und PostgreSQL, ein "starkes, quelloffenes objektrelationales Datenbankmanagementsystem mit über 30 Jahre aktiver Entwicklung", welche beide unterschiedliche Vorteile haben. PostgreSQL steht dabei repräsentativ für weitere SQL-basierte, objektrelationale Datenbankmanagesysteme, welche klassisch in der Industrie viel verwendet werden und sich der Beliebtheit der Programmierer-Gemeinschaft erfreuen.

% TODO postgres
\paragraph{}
Wie der Name vermuten lässt, ist PostgreSQL ein SQL-basiertes Datenbankmanagementsystem, welches Daten in normalisierten Tabellen verwaltet. Tabellen werden über Foreign Keys referenziert und können über Joins in Datenbankabfragen verbunden werden. PostgreSQL ist in der Lage, große Datenmengen effizient zu speichern und zu filtern.

\paragraph{}
MongoDB wird von großen Firmen wie ebay, Google, EA, coinbase und SAP verwendet.

Als Dokumentdatenbank speichert MongoDB Daten in dem BSON-Format, einem JSON-ähnlichen Dokumentformat. Dies erhöht die Lesbarkeit und ist aussagekräftiger als das herkömmliche Zeilen/Spaltenmodell, welches man unter anderem bei ORDBMS wie Postgres findet. https://www.mongodb.com/de-de

\textbf{Vorteile}\\
Vorteile MongoDB: 

\textbf{Skalierbarkeit}\\
Skalierbarkeit: Sollte unsere Webseite in Zukunft viel Datenverkehr generieren, ist es einfach, die Kapazitäten der Datenbank zu erhöhen. Dies gibt uns Sicherheit und verringert das Risiko.
Flexibilität:

\textbf{BSON}\\
MongoDB speichert Daten im "binary JSON"-Format. JSON wiederum steht für "JavaScript Object Notation". Wir verwenden in unserem Techstack ausschließlich JavaScript. Ein Dateiformat in der Datenbank zu verwenden, welches nativ von JavaScript herkommt, hat einige Vorteile: Das Schreiben der Schnittstellen ist einfacher, da die Daten nicht umgewandelt werden müssen. Das JSON-Format an sich ist dafür konzipiert, sowohl für Menschen als auch für Maschinen lesbar zu sein. Auch muss kein weiteres Dateinformat gelernt werden, da das JavaScript-Entwicklerteam bereits mit dem Format vertraut ist.

\textbf{Atlas}\\
MongoDB Inc., die Entwickler der Datenbank, bieten mit MongoDB Atlas eine database-as-a-service Lösung an, die für kleine Datenbanken sogar kostenlos ist. Dies erleichtert die Arbeit um einiges und ermöglicht uns, die Zeit effizienter dafür zu nutzen, das Projekt weiterzubringen. Sollte unser Projekt erfolgreich sein und in kurzer Zeit viele Nutzer generieren, ist es verhältnismäßig einfach, mehr Cloud-Kapazitäten zu buchen.

\textbf{Denormalisierung}\\
Denormalisierte Datenbank: Während objektrelationale Datenbankmanagesysteme meist nach dem Prinzip der Normalformen normalisiert werden, setzt MongoDB an einigen Stellen auf Denormalisierung für bessere Performance. So ist es neben der klassischen Referenz per ID möglich, einzelne Dokumente oder ganze Arrays in ein Dokument einzubetten. Dies sorgt dafür, dass die Schreibgeschwindigkeit abnimmt, da an mehreren Stellen Daten verändert werden müssen, dafür aber die Lesegeschwindigkeit oft zunimmt - Die Daten befinden sich alle in einem Dokument und müssen nicht durch Joins aus verschiedenen Tabellen zusammengesucht werden.
Auf einer Online-Plattform wie unserer werden oft Daten gelesen, aber selten geändert oder hinzugefügt, daher eignet sich für die meisten Daten Denormalisierung.
Auch bei PostgreSQL ist eine Denormalisierung möglich, jedoch glauben wir, dass MongoDB besser dafür ausgelegt ist.

\textbf{Beliebtheit}\\
Mit X und Y gehört MongoDB zu einer der beliebtesten Datenbanken. Beliebte npm-Pakete wie "mongoose" erleichtern die Verwendung von MongoDB weiter und das npm-Paket "graphql-compose-mongoose" ermöglicht eine autogenerierte, funktionsfähige Schnittstelle durch GraphQL.

\textbf{Erfahrung}\\
Das Entwicklerteam hat in der Vergangenheit bereits Erfahrung in MongoDB sammeln können und ist gut mit der Datenbank zurecht gekommen. 

Sowohl PostgreSQL als auch MongoDB eigenen sich sehr gut für unser Projekt. Uns ist bei der Recherche jedoch aufgefallen, dass MongoDB in unserem verwendeten Tech-Stack mit Express, React, und Node (MERN) deutlich mehr Verwendung findet als die Alternative mit PostgreSQL. Sollte
