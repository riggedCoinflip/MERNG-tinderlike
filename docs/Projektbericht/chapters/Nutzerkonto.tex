\paragraph{}
Wenn ein Besucher ein Konto erstellt, müssen entsprechende Daten erstellt werden. Diese Daten werden in der Datenbank gespeichert, um auch in Zukunft drauf zugreifen zu können. Als Datenbank haben wir uns für MongoDB entschieden, da <GRÜNDE>.

Im Nutzerkonto-Dokument sind etliche Daten angegeben:

\begin{center}
    \begin{tabular}{ |p{0.15\linewidth}|p{0.12\linewidth}|p{0.64\linewidth}| } 
     \hline
     Feld & Öffentlich? & Beschreibung \\ 
     \hline
      id & ja & einzigartige ID, um den Nutzer bestimmen zu können. Automatisch generiert.\\
      Name & ja & einzigartiger Name, mit dem der Nutzer anderen Nutzern angezeigt wird \\
      normalisierter Name & nein & Name in Kleinbuchstaben. Wird verwendet, um die Einzigartigkeit von Namen zu gewährleisten \\ 
      Email & nein & private eMail des Nutzers \\
      Rolle & nein & Gibt an, ob der Nutzer autorisiert ist - Moderatoren und Administratorkonten haben mehr Rechte \\ 
      Geburtsdatum & nein & privates Geburtsdatum des Nutzers \\ 
      Alter & ja & öffentliches Alter des Nutzers \\ 
       Sprachen & ja & Liste von Sprachen, die der Nutzer sprechen kann \\
       Geschlecht & ja & gesellschaftliches Geschlecht des Nutzers. \\ 
       Spielposition & ja & Position, die der Spieler in League of Legends am liebsten einnimmt. Der Nutzer kann seine zwei Lieblingspositionen angeben \\ 
       Freitext & ja & kurzer Text, in dem der Nutzer sich beschreiben kann. \\
       Avatar & ja & URI [GLOSSAR!] vom Avatarbild des Nutzers \\ 
        Freunde & nein & Liste von allen Freunden des Nutzers. Beinhaltet die NutzerID und die ChatID \\ 
        Geblockt & nein & Liste von NutzerIDs der Nutzer, die geblockt wurden \\ 

     \hline
    \end{tabular}
\end{center}

Alle Daten bis auf die ID und den Namen sind freiwillige Angaben. Wir möchten dem Nutzer die Entscheidung geben, welche Daten er uns und den anderen Nutzern preisgeben will.

Um die privaten Daten der Nutzer möglichst zu schützen, haben wir uns dazu entschieden, das Alter öffentlich zu halten, während das Geburtsdatum privat bleiben muss. Dies hat uns vor eine technische Herausforderung gestellt.
Im ersten Versuch entschieden wir uns dafür, das Alter als virtuelles Feld zu definieren. Virtuelle Felder existieren nicht persistent auf der Datenbank, sondern werden erst bei Anfrage des Dokumentes berechnet. Dies hat leider den entscheidenden Nachteil, dass bei jeder Abfrage auch das Geburtsdatum in Erfahrung gebracht werden muss. Wir haben keine möglichkeit gefunden, in GraphQL zu definieren, dass das Geburtsdatum abgefragt werden soll, aber nicht als öffentliches Feld bei der Abfrage zur verfügung steht. Ein weiterer großer Nachteil ist, dass, dadurch dass das Alter nicht persistent ist, nicht nach diesem gefiltert werden kann. Eine Abfrage wie "zeige mir alle Nutzer an, die zwischen X und Y Jahre alt sind" ist nicht möglich, stattdessen müsste die Abfrage heißen: "zeige mir alle Nutzer an, die zwischen Datum 1 und Datum 2 geboren wurden". Damit wäre das Geburtsdatum wiederum nicht privat.
Als zweites haben wir versucht, das Alter als Getter des Geburtsdatums zu definieren - wenn das Geburtsdatum erfragt wird, erhält die abfragende Partei stattdessen das berechnete Alter. Dies löst das Problem des öffentlichen Geburtsdatums zwar, lässt uns aber immer noch nicht nach dem Alter filtern.
Im Endeffekt haben wir uns dafür entschieden, das Alter als persistentes Feld anzulegen. Das Alter wird durch einen CRON-Job jeden Tag für jeden Nutzer einmal neu berechnet und ist damit tagesaktuell. Durch das persistente Alters-Feld ist es möglich, nach diesem zu filtern. Die benötigte Rechenkapazität, um täglich das Alter zu erneuern, ist vernachlässigbar. Sollte das Geburtsdatum geändert werden, wird das Alter zudem neu berechnet.

Wenn man ein Profil bis zu ein Jahr lang verfolgt, ist es immer noch möglich, das Geburtsdatum anhand der Änderung des Alters am Geburtstag herauszubekommen, allerdings glauben wir, dass dieses Risiko zu gering ist, um ernsthaft Schaden anzurichten. Das Interesse der Nutzer, zu wissen, wie alt ihr gegenüber ist, überwiegt das Risiko, dass der Geburtstag von einzelnen Nutzern in Erfahrung gebracht wird.

<TODO> Geschlecht <muss weiter erklärt werden>

<TODO> Um weitere Statistiken anzeigen zu können, ist es dem Nutzer möglich, sein Konto via Riot-Schnittstelle mit seinen "League of Legends"-Konto zu verbinden. Dies ermöglicht es, das Level, die Elo, die Lieblingshelden nd viele weitere Statistiken anzuzeigen.