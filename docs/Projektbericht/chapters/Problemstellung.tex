% Was ist das Problem
\paragraph{}
Unsere Aufgabe ist es, eine Webplattform für Spieler des Videospiels "League of Legends" zu schaffen, auf der sich Spieler kennenlernen und zu einer gemeinsamen Partie verabreden können. 

Um eine möglichst gute Plattform zu bieten, (sind verschiedene Schritte zu erledigen) Carlo: sind folgende Schritte zu erledigen:
\begin{itemize}
    \item Erstellung eines Canban-Dashboards mit einem Projektmanagementstool für die Verwaltung der Aufgaben
    \item Einrichtung eines Repository für die Quellcodeverwaltung 
    \item Konzeption und Entwicklung einer Datenbank
    \item Entwicklung einer Schnittstelle zur Programmierung von Anwendungen (API)
    \item Entwicklung einer Benutzeroberfläche
  \end{itemize}
\subsection*{Aufmerksamkeit/Marketing}
\addcontentsline{toc}{subsection}{Aufmerksamkeit/Marketing} 
\paragraph{}
Im Rahmen der Projektarbeit steht Marketing nicht im Fokus. Wir beschreiben daher nur kurz, worauf zu achten ist, wenn das Produkt nach dem Praxisprojekt auf den Markt kommen soll.

\paragraph{}
Wie bei sozialen Plattformen und sozialen Netzwerken üblich, erhöht sich der Nutzen durch positive Netzwerkeffekte mit steigender Nutzeranzahl. Ein Ziel ist es daher, eine möglichst große Nutzerbasis aufzubauen, die unsere Webseite möglichst oft verwendet.
Zuerst müssen wir Mögichkeiten finden, damit Personen von unserer Webseite überhaupt erfahren. Auf technischer Seite ist dies mit Suchmaschinenoptimierung möglich, Marketingtechnisch haben wir verschiedene Möglichkeiten der Werbung, unter anderem Bannerwerbung, Influencermarketing und "Kunden werben Kunden".

Sobald Personen unsere Webseite besuchen, müssen diese überzeugt werden, dass unser Produkt gut ist und dass sie ein Konto erstellen sollten. Es sollte daher auf einer ansprechenden Startseite das Produkt vorgestellt werden, auf dem erklärt wird, was das Ziel unserer Plattform ist und wie wir dieses Ziel erreichen wollen.

Je besser unser Produkt ist, desto wahrscheinlicher ist es, dass ein Nutzer durch intrinsische Motivation unsere Webseite verwendet. Es ist daher wichtig, ein Produkt mit möglichst hoher Qualität zu erstellen. Ein Nutzer, der von unserer Webseite begeistert ist, wird wahrscheinlicher die Webseite öfter verwenden und Freunden von der Webseite erzählen, was wiederum neue Nutzer generieren kann.

\subsection*{Das Nutzerkonto}
\addcontentsline{toc}{subsection}{Das Nutzerkonto} 

\paragraph{}
Um die Webseite vollständig nutzen zu können, sollen Besucher der Seite ein Konto erstellen. Dazu sollen sie eine E-Mail angeben sowie ein Passwort und einen freigewählten Nutzernamen aussuchen. Auch sollen sie nach Kontoerstellung Nutzerdaten wie ein Profilbild, das Alter und einen Freitext angeben können. Die angegeben Daten sollen dem Nutzer dabei helfen, Kontakte auf der Plattform zu finden und können als Gesprächsthema dienen.

Auch ist geplant, dass das Nutzerkonto mit dem Riot-Konto verbunden werden kann. Dies soll es ermöglichen, dass Spielstatistiken, wie zum Beispiel die meist gespielten Champions, die Lieblingsposition und die Elo, angezeigt werden. Ein Nutzer, der auf Kontaktsuche ist, erhält damit mehr Informationen über den Spieler und kann somit besser abschätzen, ob er Kontakt aufnehmen will. 

\subsection*{Kontaktsuche}
\addcontentsline{toc}{subsection}{Kontaktsuche} 

\paragraph{}
Um mit anderen Nutzern in Kontakt treten zu können, soll es möglich sein, andere, zufällige Nutzerprofile anzuzeigen. Profile sollen einzeln hintereinander angezeigt werden; nachdem sich der Nutzer entschieden hat, ob er dem angezeigten Nutzer einen Like [GLOSSAR!] vergeben will, wird das nächste Profil angezeigt.

Sobald zwei Nutzer gegenseitig einen Like vergeben haben, sollen diese befreundet sein. Befreundete Nutzer sollen untereinander Nachrichten austauschen können.

\subsection*{Filter}
\addcontentsline{toc}{subsection}{Filter} 

\paragraph{}
Um die Qualität der angezeigten Profile zu erhöhen und eine möglichst gute Nutzererfahrung zu gewährleisten, soll es dem Nutzer möglich sein, seine Suche einzugrenzen. Durch Filter werden ihm nur die Nutzer angezeigt, die für ihn relevant sind.

\subsection*{Regeln und Richtlinien}
\addcontentsline{toc}{subsection}{Regeln und Richtlinien} 

\paragraph{}
Die Plattform sollte einer sichere Ort sein, auf dem die Spieler unabhängig ihrer Eigenschaften respektiert werden. Wir sind der Meinung, dass die meisten Nutzer in der Lage sind, respektvoll miteinander umzugehen.
Dies können wir jedoch nicht gewährleisten.

Es wird das Recht vorbehalten, Nutzern, die gegen unsere Regeln und Richtlinien verstoßen, den Zugang zu unserer Plattform zu verwehren. Sollte sich ein Nutzer von einem anderen Nutzer beleidigt fühlen oder andersweitig der Meinung sein, dass dieser Nutzer gegen die Regeln verstößt, soll dieser Regelbrecher gemeldet werden können. Wir als Betreiber der Plattform sollen in der Lage sein, die Meldungen zu verarbeiten und angemessen mit den Regelbrechern zu verfahren.

Unabhängig davon soll es den Nutzern möglich sein, andere Nutzer zu blockieren. Blockierte Nutzer sollen dem aktiven Nutzer nicht mehr in der Spielersuche angezeigt werden, nicht mehr Nachrichten verschicken und aktive Chats sollen aufgelöst werden. Einen anderen Spieler zu blockieren heißt nicht zwangsläufig, dass dieser gegen Regeln oder Richtlinien verstößt, es kann auch sein, dass der Nutzer einfach den Kontakt abbrechen will. Sollte ein Nutzer jedoch gemeldet werden, soll er automatisch blockiert werden.

\subsection*{Internationalisierung}
\addcontentsline{toc}{subsection}{Internationalisierung} 
Die Webseite soll in Englisch angeboten werden, da dies eine der weit verbreitetsten Sprachen der Welt ist und damit viele Nutzer erreicht.

\subsection*{Einstellungen}
\addcontentsline{toc}{subsection}{Einstellungen} 
Um den verschiedenen Vorlieben unterschiedlichster Nutzer gerecht zu werden, sollen diese in der Lage sein, verschiedene Optionen in den Einstellungen zu verwalten. Unter anderem sollen Nutzer in der Lage sein, auszusuchen, wie und ob sie über neue Nachrichten, Freundschaften und Neuigkeiten informiert werden.

\subsection*{Nachrichten}
\addcontentsline{toc}{subsection}{Nachrichten} 
Sobald zwei Nutzer befreundet sind, können sie sich gegenseitig Nachrichten schreiben. Dazu soll auf einer dafür eingerichteten Seite in ein Eingabefeld der zu versendende Text eingegeben werden. Sobald die Nachricht abgeschickt wird, soll der andere Nutzer möglichst Zeitnah von dieser erfahren. 

Zeichen sollen nach dem UTF-8 Zeichensatz erlaubt sein.

\subsection*{Sicherheit}
\addcontentsline{toc}{subsection}{Sicherheit} 
Die Sicherheit der Nutzerdaten ist von hoher Priorität. Passwörter müssen besonders geschützt und nach derzeitigem kryptologischen Stand der Technik gesichert werden, um den Zugriff von unautorisierten Angreifern zu unterbinden. Identifizierende Daten wie E-Mail-Adressen dürfen nicht öffentlich einsehbar sein.

Sollte in einem späteren Schritt die Plattform veröffentlicht werden, ist auf DSGVO-Konformität zu achten.

\subsection*{Dokumentation}
\addcontentsline{toc}{subsection}{Dokumentation} 
"Technische Schulden" (en. "technical dept", schlechte Umsetzung von Software, die einen erheblichen Mehraufwand in der Zukunft bedeutet) sorgen in vielen Fällen für Probleme im Laufe des Produktlebenszyklus. Um die Wahrscheinlichkeit zu verringern, dass sich technische Schulden anhäufen und, soll eine gut leserliche Dokumentation zum Quellcode geschrieben werden, welche es allen Gruppenmitgliedern ermöglichen soll, den überblick zu halten, welcher Codeschnipsel welches Problem löst. Desweiteren sollen umfangreiche Tests durchgeführt werden, welche die Qualität des Codes gewährleisten sollen.

Um die Kommunikation zwischen Backend und Frontend zu gewährleisten, sollen Schnittstellen entsprechend dokumentiert werden. Es sollte einem Frontend-Entwickler möglich sein, nur mit der Dokumentation auf die Schnittstelle zuzugreifen und die Informationen zu erhalten, die er benötigt, um entsprechende Daten im Frontend anzeigen zu können.

\subsection*{Konto löschen}
\addcontentsline{toc}{subsection}{Konto löschen} 

Es wird Benutzer geben, die aus verschiedensten Gründen unsere Webseite nicht weiter verwenden wollen und ihr Konto löschen wollen. Unsere Aufgabe ist es, eine sichere Löschung des Benutzerkontos zu gewährleisten und persönliche Daten nach Datenschutzvorschriften aus der Datenbank zu entfernen.

Optional können wir zudem den Nutzer darum bitten, ein Formular auszufüllen, in dem dieser uns Rückmeldung geben kann, aus welchen Gründen er die Webseite nicht weiter verwenden will.
