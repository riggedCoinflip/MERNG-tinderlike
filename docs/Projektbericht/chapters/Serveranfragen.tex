\subsection{GraphQL}
\paragraph{}
GraphQL ist eine Abfragesprache und Server-Laufzeitumgebung für APIs.
Ihre Aufgabe ist es, genau die Daten zu liefern, die anfordert werden, und nicht mehr.
\\
Mit GraphQL sind APIs schnell, flexibel und einfach für Entwickler.
\\ \\
Laut dem 2020 State of the API Report von Postman.com steht GrapQL an fünfter Stelle der spannendsten Technologien für 2021.
% \begin{quote}  {Vgl. u.a. \cite{RH1}} \end{quote}

Im Hinblick auf die Art und Weise, wie Abfragen an den Server mit Hilfe von \\ GraphQL behandelt werden können, sind folgende Aspekte zu beachten.
\\
\begin{quote}
    \textbf{Vorteile}\\

    \begin{itemize}
        \item
              GraphQL-Aufrufe werden in einem einzigen Round Trip gehandhabt. Wir bekommen genau die Daten, die angefragt haben (kein Over-Fetching).

        \item
              Stark definierte Datentypen verringern das Risiko einer Fehlkommunikation zwischen Client und Server.

        \item
              GraphQL ist introspektiv. So können wir eine Liste der verfügbaren Datentypen anfordern. Dies ist ideal für automatisch erstellte Dokumente.

        \item
              Eine Anwendungs-API kann sich mit GraphQL weiterentwickeln, ohne dass bestehende Anfragen beeinträchtigt werden.
        \item
              GraphQL schreibt keine spezifische Anwendungsarchitektur vor. Es kann auf einer vorhandenen REST-API installiert und mit aktuellen API-Management-Tools verwendet werden.
        \item
              Als Alternative zu REST ermöglicht GraphQL Entwicklern die Erstellung von Abfragen zur Extraktion von Daten aus mehreren Quellen mit einem einzigen API-Abfrage.

    \end{itemize}
   
    \textbf{Nachteile}
    \begin{itemize}
        \item
        Für Entwickler, die sich bereits mit REST-APIs auskennen, bedeutet GraphQL weiteren Lernaufwand.
     \item 
     Mit GraphQL verschiebt sich die Funktionalität von Datenabfragen zur Serverseite, was zusätzliche Komplexität für Serverentwickler bedeutet.

\end{itemize}

    \footnote{Vgl. u.a. \cite{RH1}}
\end{quote}

\textbf{Wie wurden die Anfragen in unserem Projekt verwaltet?}\\
Alle Abfragen und Mutations wurden in einem Ordner gesammelt.
Diese wurden für die spätere Verwendung in den React-Komponenten exportiert.

\begin{lstlisting}
    export const GET_MY_INFO = gql`
    {
      userSelf {
        _id
        name
        aboutMe
        languages
        gender
        avatar
        ingameRole
        dateOfBirth
        friends { user chat }        
        blocked
      }
    }
  `
//Beispiel einer Abfrage
\end{lstlisting}


\newpage
\textbf{Apollo Client}\\
%Warum haben wir uns dafür entschieden?
Nachdem eine Abfrage exportiert wurde, ist sie bereit, in einer React-Komponente importiert \\ und angewendet zu werden.

\begin{lstlisting}
import { GET_MY_INFO } from "./GraphQL/Queries"
import { useQuery } from "@apollo/client"

const { loading, error, data } = useQuery(
GET_MY_INFO,
ContextHeader(token),
)
//Code-Auszug in frontend/src/App.js

\end{lstlisting}
Die Konstante ContextHeader enthält das Token in der Struktur, die erforderlich ist, um die Abfrage nur dann stellen zu können, wenn der Benutzer dazu berechtigt ist.
\\
Sollte das Token undefined, null oder ungültig sein, wird der Server ein Fehler zurückgegeben.
\\ 

Der useQuery Hook liefert ein Ergebnisobjekt, das Folgendes enthält.
\\
\textbf{loading:}\\
Ein boolescher Wert, der angibt, ob die Abfrage in Bearbeitung ist.
Wenn loading true ist, ist die Anfrage noch nicht abgeschlossen. Typischerweise kann diese Information verwendet werden, um einen Lade-Spinner anzuzeigen.
\\

\textbf{error:}\\
Ein Laufzeitfehler mit den Eigenschaften von GraphQL Errors und network Error.
Dieses enthält Informationen darüber, was bei der Abfrage schief gelaufen ist.
\\

\textbf{date:}\\
Ein Objekt, das das Ergebnis der GraphQL-Abfrage enthält.
\\Es enthält die tatsächlichen Daten vom Server.
\\

AUCH MUTATIONS ZEIGEN\dots
\newpage
\subsection{Axios}
Zusätzlich zu den GraphQL-Abfragen, wurde eine Post-Anfrage mit Axios bereitgestellt.
Mit dieser war es möglich, Bilder auf die S3 Speicher von AWS hochzuladen.

\begin{quote}
Axios ist ein Promise-basierter HTTP-Client für node.js und den browser. Er ist isomorphisch (= kann auf dem server und im browser verwendet weden). Auf der Server-Seite wir das modul http verwendet, während im Browser XMLHttpRequests (ajax) ausgeführt werden.
\end{quote}

\begin{lstlisting}
    function disableBtn() {
        const uploadBtn = document.getElementById("uploadBtn")
        uploadBtn.disabled = true
        uploadBtn.style.background = "#000000"
      }
    
      function fileUploadHandler() {
        errored ? 
         disableBtn()
         /*
        Tritt ein Fehler auf, wird die Abfrage nicht gesendet und der Knopf deaktiviert 
        Ein Fehler tritt auf, wenn die Datei zu gross ist oder oder wenn sie ein ungueltiges  Format hat
        */
        :
        console.log("uploading pic...", file?.name)
        const fd = new FormData()
        /* avatar ist der Name der Datei, die hochgeladen wird
          file ist der zu sendende  Wert */
        fd.append("avatar", file)  
    
        axios
          .post(urlAvatar, fd, {
            headers: {
              "x-auth-token": TOKEN,
            },
          })
          .then((res) => {
            setState((state) => ({ ...state, avatar: res?.data?.location }))    
            //In location finden wir die URL fuer das gerade hochgeladende Bild 
          })
      }\end{lstlisting}

Das HTML-Eingabefeld "input" wurde folgendermaße definiert.
\begin{lstlisting}
    <input type="file" onChange={fileSelectedHandler} />
\end{lstlisting}


Das Format der hochzuladenden Dateien wurde limitiert, damit nur zulässige Dateien an den Server gesendet werden.
\begin{lstlisting}
    const admittedImageFormats = ["png", "jpg", "jpeg"]
\end{lstlisting}

Die Größe der hochzuladenden Datei wurde um 1Mb abgegrenzt.
Durch die Eigenschaft „size“ der ausgewählten Datei konnten wir auf die Größe der Datei zugreifen.
\begin{lstlisting}
    const imageSize = e.target.files[0].size
\end{lstlisting}