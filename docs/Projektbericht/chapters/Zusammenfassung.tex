%Es hat sich gezeigt, dass mit modernen JavaScript-Entwicklungswerkzeugen 
%auch ohne umfassende Kenntnisse der Softwareentwicklung zugänglich sind.
Es hat sich gezeigt, wie die Entwicklung einer Webanwendung nach dem MERN-Stack erfolgt. Dies in einem Zeitraum von Mai 2021 bis Mitte August 2021. Mit grundlegenden Programmierkenntnissen war es möglich, eine funktionelle Anwendung mit folgenden Funktionen zu entwicklen; die Registrierung für neue Benutzer, die Anmeldung für bestehende Benutzer, die Verwaltung deren persönlichen Daten, die Interaktion mit anderen Benutzern auf der Grundlage ihrer Präferenzen und einen auf Textnachrichten basierenden Kommunikationskanal.
\\\\
%TInfra
Kapitel \ref{kap_Technologieinfrastruktur}
\\\\
%DB
Kapitel \ref{kap_Datenbank} :Die Benutzerdaten wurden permanent in einer nicht-relationalen Datenbank gespeichert.
\\\\
%Backend
Kapitel \ref{kap_Backend} 
\\\\
%Schnittstelle 
Kapitel \ref{kap_Schnittstelle}\\\\
%Frontend
Kapitel \ref{kap_Frontend}
%Qualitätssicherung
Die Erstellung von den End-To-End-Tests beschrieben im Kapitel \ref{kap_QS} würde die Testzeit für komplexere Anwendungen beschleunigen. Im Falle des vorliegenden Projekts handelte es sich um ein zusätzlicher Werkzeug, dessen Einbindung in den Entwicklungszyklus etwa 5 Tage in Anspruch nahm.
%Was konnte nicht geschafft werden?