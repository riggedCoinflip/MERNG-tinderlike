%Es hat sich gezeigt, dass mit modernen JavaScript-Entwicklungswerkzeugen 
%auch ohne umfassende Kenntnisse der Softwareentwicklung zugänglich sind.
Es hat sich gezeigt, wie die Entwicklung einer Webanwendung nach dem MERN-Stack erfolgt. Dies in einem Zeitraum von Mai 2021 bis Mitte August 2021. Mit grundlegenden Programmierkenntnissen war es möglich, eine funktionelle Anwendung mit folgenden Funktionen zu entwicklen; die Registrierung für neue Benutzer, die Anmeldung für bestehende Benutzer, die Verwaltung deren persönlichen Daten, die Interaktion mit anderen Benutzern auf der Grundlage ihrer Präferenzen und einen auf Textnachrichten basierenden Kommunikationskanal.
\\\\
%TInfra
In Kapitel \ref{kap_Technologieinfrastruktur} wurde die Architektur der Anwendung kurz eingeführt. %DB
Die Auswahl der Datenbank und deren Schemata wurde in Kapitel \ref{kap_Datenbank} erläutert. In Kapitel \ref{kap_Backend} und \ref{kap_Schnittstelle} befindet sich die Logik, um die Daten der Datenbank zu manipulieren. An dieser Stelle wurde erklärt, wie die REST-Schnittstelle für das Profilbild im Amazon S3 und die GraphQL-Endpunkte für die Verwaltung der Nutzerdaten entwickelt wurde.  
%Backend %Schnittstelle 
%Frontend
Das Kapitel \ref{kap_Frontend} zeigt welche Kriterien für die Wahl von React berücksichtigt wurden. Außerdem, wurde die Entwicklung der Benutzeroberfläche mit React-Hooks gezeigt. 
%Qualitätssicherung
Die Erstellung von den Unit- und End-To-End-Tests beschrieben im Kapitel \ref{kap_QS} würde die Testzeit für komplexere Anwendungen beschleunigen. Im Falle des vorliegenden Projekts handelte es sich um ein zusätzlicher Werkzeug, dessen Einbindung in den Entwicklungszyklus etwa fünf Tage in Anspruch nahm. %Einer signifikante Wert haben die Unit-Test, 
%Was konnte nicht geschafft werden?
\\\\
Der Fokus lag daran, eine funktionelle Anwendung zu entwicklen. Es hat sich auf Funktionen verzichtet, welche die geplante Zeitraum des Projekts überschritten. Unter anderem die Anmeldung über SSO mit zum Beispiel eine Facebook- oder Google-Konto. %Für einen späteren Zeitpunkt wird sich dieses Projekt bei WIE HEISST DAS?
Eine bessere Gebrauchlichkeit %Usability 
bietet für die vorliegenden Applikation, eine mobile Applikation. Deshalb ist es sinnvoll eine weitere Version der Anwendung mit React-Native umzuwandeln.
%Kontovalidierung fehlt mit der API-Riot

\section{Arbeitsverteilung}

